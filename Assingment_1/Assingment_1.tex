\documentclass[journal,12pt,twocolumn]{IEEEtran}

\usepackage{setspace}
\usepackage{gensymb}
\singlespacing
\usepackage[cmex10]{amsmath}
\usepackage{tabularx}

\usepackage{amsthm}
\usepackage{tikz-qtree}
\usepackage[utf8]{inputenc}

\usepackage{mathrsfs}
\usepackage{txfonts}
\usepackage{stfloats}
\usepackage{bm}
\usepackage{cite}
\usepackage{cases}
\usepackage{subfig}

\usepackage{longtable}
\usepackage{multirow}

\usepackage{enumitem}
\usepackage{mathtools}
\usepackage{steinmetz}
\usepackage{tikz}
\usepackage{circuitikz}
\usepackage{verbatim}
\usepackage{tfrupee}
\usepackage[breaklinks=true]{hyperref}
\usepackage{graphicx}
\usepackage{tkz-euclide}

\usetikzlibrary{calc,math}
\usepackage{listings}
    \usepackage{color}                                            %%
    \usepackage{array}                                            %%
    \usepackage{longtable}                                        %%
    \usepackage{calc}                                             %%
    \usepackage{multirow}                                         %%
    \usepackage{hhline}                                           %%
    \usepackage{ifthen}                                           %%
    \usepackage{lscape}     
\usepackage{multicol}
\usepackage{chngcntr}

\DeclareMathOperator*{\Res}{Res}

\renewcommand\thesection{\arabic{section}}
\renewcommand\thesubsection{\thesection.\arabic{subsection}}
\renewcommand\thesubsubsection{\thesubsection.\arabic{subsubsection}}

\renewcommand\thesectiondis{\arabic{section}}
\renewcommand\thesubsectiondis{\thesectiondis.\arabic{subsection}}
\renewcommand\thesubsubsectiondis{\thesubsectiondis.\arabic{subsubsection}}


\hyphenation{op-tical net-works semi-conduc-tor}
\def\inputGnumericTable{}                                 %%

\lstset{
%language=C,
frame=single, 
breaklines=true,
columns=fullflexible
}
\begin{document}


\newtheorem{theorem}{Theorem}[section]
\newtheorem{problem}{Problem}
\newtheorem{proposition}{Proposition}[section]
\newtheorem{lemma}{Lemma}[section]
\newtheorem{corollary}[theorem]{Corollary}
\newtheorem{example}{Example}[section]
\newtheorem{definition}[problem]{Definition}

\newcommand{\BEQA}{\begin{eqnarray}}
\newcommand{\EEQA}{\end{eqnarray}}
\newcommand{\define}{\stackrel{\triangle}{=}}
\bibliographystyle{IEEEtran}
\raggedbottom
\setlength{\parindent}{0pt}
\providecommand{\mbf}{\mathbf}
\providecommand{\pr}[1]{\ensuremath{\Pr\left(#1\right)}}
\providecommand{\qfunc}[1]{\ensuremath{Q\left(#1\right)}}
\providecommand{\sbrak}[1]{\ensuremath{{}\left[#1\right]}}
\providecommand{\lsbrak}[1]{\ensuremath{{}\left[#1\right.}}
\providecommand{\rsbrak}[1]{\ensuremath{{}\left.#1\right]}}
\providecommand{\brak}[1]{\ensuremath{\left(#1\right)}}
\providecommand{\lbrak}[1]{\ensuremath{\left(#1\right.}}
\providecommand{\rbrak}[1]{\ensuremath{\left.#1\right)}}
\providecommand{\cbrak}[1]{\ensuremath{\left\{#1\right\}}}
\providecommand{\lcbrak}[1]{\ensuremath{\left\{#1\right.}}
\providecommand{\rcbrak}[1]{\ensuremath{\left.#1\right\}}}
\theoremstyle{remark}
\newtheorem{rem}{Remark}
\newcommand{\sgn}{\mathop{\mathrm{sgn}}}
% \providecommand{\abs}[1]{\left\vert#1\right\vert}
% \providecommand{\res}[1]{\Res\displaylimits_{#1}} 
% \providecommand{\norm}[1]{\left\lVert#1\right\rVert}
% %\providecommand{\norm}[1]{\lVert#1\rVert}
% \providecommand{\mtx}[1]{\mathbf{#1}}
% \providecommand{\mean}[1]{E\left[ #1 \right]}
\providecommand{\fourier}{\overset{\mathcal{F}}{ \rightleftharpoons}}
%\providecommand{\hilbert}{\overset{\mathcal{H}}{ \rightleftharpoons}}
\providecommand{\system}{\overset{\mathcal{H}}{ \longleftrightarrow}}
	%\newcommand{\solution}[2]{\textbf{Solution:}{#1}}
\newcommand{\solution}{\noindent \textbf{Solution: }}
\newcommand{\cosec}{\,\text{cosec}\,}
\providecommand{\dec}[2]{\ensuremath{\overset{#1}{\underset{#2}{\gtrless}}}}
\newcommand{\myvec}[1]{\ensuremath{\begin{pmatrix}#1\end{pmatrix}}}
\newcommand{\mydet}[1]{\ensuremath{\begin{vmatrix}#1\end{vmatrix}}}
\numberwithin{equation}{subsection}
\makeatletter
\@addtoreset{figure}{problem}
\makeatother
\let\StandardTheFigure\thefigure
\let\vec\mathbf
\renewcommand{\thefigure}{\theproblem}
\def\putbox#1#2#3{\makebox[0in][l]{\makebox[#1][l]{}\raisebox{\baselineskip}[0in][0in]{\raisebox{#2}[0in][0in]{#3}}}}
     \def\rightbox#1{\makebox[0in][r]{#1}}
     \def\centbox#1{\makebox[0in]{#1}}
     \def\topbox#1{\raisebox{-\baselineskip}[0in][0in]{#1}}
     \def\midbox#1{\raisebox{-0.5\baselineskip}[0in][0in]{#1}}
\vspace{3cm}
\title{Assignment 1}
\author{Ishwari Prashad - EE18BTECH11020}
\maketitle
\newpage
\bigskip
\renewcommand{\thefigure}{\theenumi}
\renewcommand{\thetable}{\theenumi}
Download all latex-tikz and C codes from 
%
\begin{lstlisting}
https://github.com/ipsingh85/EE4013/tree/main/Assingment_1/codes
\end{lstlisting}
\begin{lstlisting}
https://github.com/ipsingh85/EE4013/tree/main/Assingment_1/Assingment_1.tex
\end{lstlisting}
\section{Problem}
(Q 5) The preOrder traversal of a binary search tree is $15$,$10$,$12$,$11$,$20$,$18$,$16$,$19$ which one of the following is postOrder traversal of the tree?

\begin{enumerate}
    \item $10$,$11$,$12$,$15$,$16$,$18$,$19$,$20$
    \item $11$,$12$,$10$,$16$,$19$,$18$,$20$,$15$
    \item $20$,$19$,$18$,$16$,$15$,$12$,$11$,$10$
    \item $19$,$11$,$18$,$20$,$11$,$12$,$10$,$15$
\end{enumerate}
\section{Definitions}
Answer : Option 2

\textbf{Binary Search Tree} is a node-based binary tree data structure which has the following properties:
\begin{enumerate}
    \item The left subtree of a node contains only nodes with keys lesser than the node’s key.
    \item The right subtree of a node contains only nodes with keys greater than the node’s key.
    \item The left and right subtree each must also be a binary search tree.
\end{enumerate}
\textbf{PreOrder traversal} 
\newline
 Algorithm preOrder
 \begin{enumerate}
     \item Visit the root.
     \item Traverse the left subtree.
     \item Traverse the right subtree.
 \end{enumerate}
 \textbf{PostOrder traversal} 
 \newline
 Algorithm Postorder
 \begin{enumerate}
     \item Traverse the left subtree.
     \item Traverse the right subtree.
     \item Visit the root.
 \end{enumerate}
\section{Explanation : } 
 so  first we will convert this preorder traversal in to a binary search tree then print the postorder traversal.
 
 as we know first element of preorder traversal is root node and element with value less than root value make left subtree and element vaule greater than root value make right subtree.
 \vspace{5mm}
 \newline
 given preOrder traversal $15$,$10$,$12$,$11$,$20$,$18$,$16$,$19$
\newline
root key      $15$
\newline
left subtree  $10$,$12$,$11$
\newline
right subtree $20$,$18$,$16$,$19$
\newline
similaraly for every left subtree and right subtree
\vspace{5mm}


\begin{tabular}{ |c|c|c|c| }
 \hline
 \multicolumn{4}{|c|}{left and right subtree for every node} \\
 \hline
 \vtop{\hbox{\strut preOrder}\hbox{\strut traversal}}&\vtop{\hbox{\strut root}\hbox{\strut node}}&\vtop{\hbox{\strut left}\hbox{\strut subtree}}&\vtop{\hbox{\strut right}\hbox{\strut subtree}}\\

 \hline
 \vtop{\hbox{\strut15 10 12 11 20 }\hbox{\strut 18 16 19}}& 15 &10 12 11& \vtop{\hbox{\strut 20 18 16 }\hbox{\strut 19}}\\
 \hline
 10 12 11&10&NA&12 11\\
 \hline
 12 11&12& 11 & NA  \\
 \hline
 20 18 16 19&20& 18 16 19  &  NA\\
 \hline
 18 16 19&18&16&19\\
 
 \hline
\end{tabular}

\vspace{5mm}
using the table draw node diagram of given preOrder traversal 
\vspace{5mm}
\setcounter{figure}{0}
\begin{figure}[!h]
\centering
\tikzset{every tree node/.style={minimum width=2em,draw,circle},
         blank/.style={draw=none},
         edge from parent/.style=
         {draw,edge from parent path={(\tikzparentnode) -- (\tikzchildnode)}},
         level distance=1.5cm}
\begin{tikzpicture}
\Tree
[.15     
    [.10 
    \edge[blank]; \node[blank]{};
    \edge[]; [.12
             \edge[];{11}
             \edge[blank]; \node[blank]{};
         ]
    ]
    [.20 
    \edge[]; [.18
             \edge[];{16}
             \edge[];{19}
             ]
    \edge[blank];\node[blank]{};
    ]
]

\end{tikzpicture}
\caption{BST diagram of given preOrder traversal} \label{Fig:M1}

\end{figure}






\vspace{10mm}


from above tree using postOrder traversal algorithm print the postOrder traversal
\newline
\textbf{ postrder traversal}
\textbf{$11$,$12$,$10$,$16$,$19$,$18$,$20$,$15$}



\end{document}
