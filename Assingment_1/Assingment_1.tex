\documentclass[journal,12pt,twocolumn]{IEEEtran}

\usepackage{setspace}
\usepackage{gensymb}
\singlespacing
\usepackage[cmex10]{amsmath}

\usepackage{amsthm}
\usepackage{tikz-qtree}

\usepackage{mathrsfs}
\usepackage{txfonts}
\usepackage{stfloats}
\usepackage{bm}
\usepackage{cite}
\usepackage{cases}
\usepackage{subfig}

\usepackage{longtable}
\usepackage{multirow}

\usepackage{enumitem}
\usepackage{mathtools}
\usepackage{steinmetz}
\usepackage{tikz}
\usepackage{circuitikz}
\usepackage{verbatim}
\usepackage{tfrupee}
\usepackage[breaklinks=true]{hyperref}
\usepackage{graphicx}
\usepackage{tkz-euclide}

\usetikzlibrary{calc,math}
\usepackage{listings}
    \usepackage{color}                                            %%
    \usepackage{array}                                            %%
    \usepackage{longtable}                                        %%
    \usepackage{calc}                                             %%
    \usepackage{multirow}                                         %%
    \usepackage{hhline}                                           %%
    \usepackage{ifthen}                                           %%
    \usepackage{lscape}     
\usepackage{multicol}
\usepackage{chngcntr}

\DeclareMathOperator*{\Res}{Res}

\renewcommand\thesection{\arabic{section}}
\renewcommand\thesubsection{\thesection.\arabic{subsection}}
\renewcommand\thesubsubsection{\thesubsection.\arabic{subsubsection}}

\renewcommand\thesectiondis{\arabic{section}}
\renewcommand\thesubsectiondis{\thesectiondis.\arabic{subsection}}
\renewcommand\thesubsubsectiondis{\thesubsectiondis.\arabic{subsubsection}}


\hyphenation{op-tical net-works semi-conduc-tor}
\def\inputGnumericTable{}                                 %%

\lstset{
%language=C,
frame=single, 
breaklines=true,
columns=fullflexible
}
\begin{document}


\newtheorem{theorem}{Theorem}[section]
\newtheorem{problem}{Problem}
\newtheorem{proposition}{Proposition}[section]
\newtheorem{lemma}{Lemma}[section]
\newtheorem{corollary}[theorem]{Corollary}
\newtheorem{example}{Example}[section]
\newtheorem{definition}[problem]{Definition}

\newcommand{\BEQA}{\begin{eqnarray}}
\newcommand{\EEQA}{\end{eqnarray}}
\newcommand{\define}{\stackrel{\triangle}{=}}
\bibliographystyle{IEEEtran}
\raggedbottom
\setlength{\parindent}{0pt}
\providecommand{\mbf}{\mathbf}
\providecommand{\pr}[1]{\ensuremath{\Pr\left(#1\right)}}
\providecommand{\qfunc}[1]{\ensuremath{Q\left(#1\right)}}
\providecommand{\sbrak}[1]{\ensuremath{{}\left[#1\right]}}
\providecommand{\lsbrak}[1]{\ensuremath{{}\left[#1\right.}}
\providecommand{\rsbrak}[1]{\ensuremath{{}\left.#1\right]}}
\providecommand{\brak}[1]{\ensuremath{\left(#1\right)}}
\providecommand{\lbrak}[1]{\ensuremath{\left(#1\right.}}
\providecommand{\rbrak}[1]{\ensuremath{\left.#1\right)}}
\providecommand{\cbrak}[1]{\ensuremath{\left\{#1\right\}}}
\providecommand{\lcbrak}[1]{\ensuremath{\left\{#1\right.}}
\providecommand{\rcbrak}[1]{\ensuremath{\left.#1\right\}}}
\theoremstyle{remark}
\newtheorem{rem}{Remark}
\newcommand{\sgn}{\mathop{\mathrm{sgn}}}
% \providecommand{\abs}[1]{\left\vert#1\right\vert}
% \providecommand{\res}[1]{\Res\displaylimits_{#1}} 
% \providecommand{\norm}[1]{\left\lVert#1\right\rVert}
% %\providecommand{\norm}[1]{\lVert#1\rVert}
% \providecommand{\mtx}[1]{\mathbf{#1}}
% \providecommand{\mean}[1]{E\left[ #1 \right]}
\providecommand{\fourier}{\overset{\mathcal{F}}{ \rightleftharpoons}}
%\providecommand{\hilbert}{\overset{\mathcal{H}}{ \rightleftharpoons}}
\providecommand{\system}{\overset{\mathcal{H}}{ \longleftrightarrow}}
	%\newcommand{\solution}[2]{\textbf{Solution:}{#1}}
\newcommand{\solution}{\noindent \textbf{Solution: }}
\newcommand{\cosec}{\,\text{cosec}\,}
\providecommand{\dec}[2]{\ensuremath{\overset{#1}{\underset{#2}{\gtrless}}}}
\newcommand{\myvec}[1]{\ensuremath{\begin{pmatrix}#1\end{pmatrix}}}
\newcommand{\mydet}[1]{\ensuremath{\begin{vmatrix}#1\end{vmatrix}}}
\numberwithin{equation}{subsection}
\makeatletter
\@addtoreset{figure}{problem}
\makeatother
\let\StandardTheFigure\thefigure
\let\vec\mathbf
\renewcommand{\thefigure}{\theproblem}
\def\putbox#1#2#3{\makebox[0in][l]{\makebox[#1][l]{}\raisebox{\baselineskip}[0in][0in]{\raisebox{#2}[0in][0in]{#3}}}}
     \def\rightbox#1{\makebox[0in][r]{#1}}
     \def\centbox#1{\makebox[0in]{#1}}
     \def\topbox#1{\raisebox{-\baselineskip}[0in][0in]{#1}}
     \def\midbox#1{\raisebox{-0.5\baselineskip}[0in][0in]{#1}}
\vspace{3cm}
\title{Assignment 1}
\author{Ishwari Prashad - EE18BTECH11020}
\maketitle
\newpage
\bigskip
\renewcommand{\thefigure}{\theenumi}
\renewcommand{\thetable}{\theenumi}
Download all latex-tikz and C codes from 
%
\begin{lstlisting}
https://github.com/ipsingh85/EE4013/tree/main/Assingment_1/codes
\end{lstlisting}
\begin{lstlisting}
https://github.com/ipsingh85/EE4013/tree/main/Assingment_1/Assingment_1.tex
\end{lstlisting}
\section{Problem}
(Q 5) The preorder traversal of a binary search tree is $15$,$10$,$12$,$11$,$20$,$18$,$16$,$19$ which one of the following is postorder traversal of the tree?

\begin{enumerate}
    \item $10$,$11$,$12$,$15$,$16$,$18$,$19$,$20$
    \item $11$,$12$,$10$,$16$,$19$,$18$,$20$,$15$
    \item $20$,$19$,$18$,$16$,$15$,$12$,$11$,$10$
    \item $19$,$11$,$18$,$20$,$11$,$12$,$10$,$15$
\end{enumerate}
\section{Solution}
Answer : Option 2

\textbf{Binary Search Tree} is a node-based binary tree data structure which has the following properties:
\begin{enumerate}
    \item The left subtree of a node contains only nodes with keys lesser than the node’s key.
    \item The right subtree of a node contains only nodes with keys greater than the node’s key.
    \item The left and right subtree each must also be a binary search tree.
\end{enumerate}
\textbf{Preorder traversal} 
\newline
 Algorithm preorder
 \begin{enumerate}
     \item Visit the root.
     \item Traverse the left subtree.
     \item Traverse the right subtree.
 \end{enumerate}
 \textbf{Postorder traversal} 
 \newline
 Algorithm Postorder
 \begin{enumerate}
     \item Traverse the left subtree.
     \item Traverse the right subtree.
     \item Visit the root.
 \end{enumerate}
 \textbf{Explanation : } 
 \newline so  first we will convert this preorder traversal in to a binary search tree then print the postorder traversal.
 \newline 
 as we know first element of preorder traversal is root node and element with value less than root value make left subtree and element vaule greater than root value make right subtree.
 \newline
 \newline given preorder traversal $15$,$10$,$12$,$11$,$20$,$18$,$16$,$19$
\newline
root key      $15$
\newline
left subtree  $10$,$12$,$11$
\newline
right subtree $20$,$18$,$16$,$19$
\newline
similaraly for every left subtree and right subtree
\vspace{5mm}


\begin{tabular}{ |p{4cm}||p{0.8cm}|p{2cm}|p{2cm}|  }
 \hline
 \multicolumn{4}{|c|}{Country List} \\
 \hline
 preorder traversal& root node&left subtree&right subtree\\
 \hline
 $15$,$10$,$12$,$11$,$20$,$18$,$16$,$19$ & $15$ &$10$,$12$,$11$&$20$,$18$,$16$,$19$ \\
 $10$,$12$,$11$&$10$&No element less than $10$&$12$,$11$\\
 $12$,$11$&$12$& $11$ & No element greater than $12$ \\
 $20$,$18$,$16$,$19$&$20$&  $18$,$16$,$19$ &  NA\\
 $18$,$16$,$19$&   $18$  &$16$&$19$\\
 
 \hline
\end{tabular}

\newline
\vspace{5mm}
using the table create a BST 
\vspace{5mm}


\tikzset{every tree node/.style={minimum width=2em,draw,circle},
         blank/.style={draw=none},
         edge from parent/.style=
         {draw,edge from parent path={(\tikzparentnode) -- (\tikzchildnode)}},
         level distance=1.5cm}
\begin{tikzpicture}
\Tree
[.15     
    [.10 
    \edge[blank]; \node[blank]{};
    \edge[]; [.12
             \edge[];{11}
             \edge[blank]; \node[blank]{};
         ]
    ]
    [.20 
    \edge[]; [.18
             \edge[];{16}
             \edge[];{19}
             ]
    \edge[blank];\node[blank]{};
    ]
]
\end{tikzpicture}
\vspace{10mm}
\newline

from above tree using postorder traversal algorithm print the postorder traversal
\newline
\textbf{hence postorder traversal}
\textbf{$11$,$12$,$10$,$16$,$19$,$18$,$20$,$15$}



\end{document}
